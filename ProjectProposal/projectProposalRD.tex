%% bare_jrnl_compsoc.tex
%% V1.4a
%% 2014/09/17
%% by Michael Shell
%% See:
%% http://www.michaelshell.org/
%% for current contact information.
%%
%% This is a skeleton file demonstrating the use of IEEEtran.cls
%% (requires IEEEtran.cls version 1.8a or later) with an IEEE
%% Computer Society journal paper.
%%
%% Support sites:
%% http://www.michaelshell.org/tex/ieeetran/
%% http://www.ctan.org/tex-archive/macros/latex/contrib/IEEEtran/
%% and
%% http://www.ieee.org/

%%*************************************************************************
%% Legal Notice:
%% This code is offered as-is without any warranty either expressed or
%% implied; without even the implied warranty of MERCHANTABILITY or
%% FITNESS FOR A PARTICULAR PURPOSE! 
%% User assumes all risk.
%% In no event shall IEEE or any contributor to this code be liable for
%% any damages or losses, including, but not limited to, incidental,
%% consequential, or any other damages, resulting from the use or misuse
%% of any information contained here.
%%
%% All comments are the opinions of their respective authors and are not
%% necessarily endorsed by the IEEE.
%%
%% This work is distributed under the LaTeX Project Public License (LPPL)
%% ( http://www.latex-project.org/ ) version 1.3, and may be freely used,
%% distributed and modified. A copy of the LPPL, version 1.3, is included
%% in the base LaTeX documentation of all distributions of LaTeX released
%% 2003/12/01 or later.
%% Retain all contribution notices and credits.
%% ** Modified files should be clearly indicated as such, including  **
%% ** renaming them and changing author support contact information. **
%%
%% File list of work: IEEEtran.cls, IEEEtran_HOWTO.pdf, bare_adv.tex,
%%                    bare_conf.tex, bare_jrnl.tex, bare_conf_compsoc.tex,
%%                    bare_jrnl_compsoc.tex, bare_jrnl_transmag.tex
%%*************************************************************************


% *** Authors should verify (and, if needed, correct) their LaTeX system  ***
% *** with the testflow diagnostic prior to trusting their LaTeX platform ***
% *** with production work. IEEE's font choices and paper sizes can       ***
% *** trigger bugs that do not appear when using other class files.       ***                          ***
% The testflow support page is at:
% http://www.michaelshell.org/tex/testflow/


\documentclass[10pt,conference,onecolumn,compsoc]{IEEEtran}


\usepackage{hyperref}
\usepackage{enumitem}
\setlist[itemize]{leftmargin=3 cm}
\setlist[enumerate]{leftmargin=3cm}



% *** CITATION PACKAGES ***
%
\ifCLASSOPTIONcompsoc
  % IEEE Computer Society needs nocompress option
  % requires cite.sty v4.0 or later (November 2003)
  \usepackage[nocompress]{cite}
\else
  % normal IEEE
  \usepackage{cite}
\fi
% cite.sty was written by Donald Arseneau
% V1.6 and later of IEEEtran pre-defines the format of the cite.sty package
% \cite{} output to follow that of IEEE. Loading the cite package will
% result in citation numbers being automatically sorted and properly
% "compressed/ranged". e.g., [1], [9], [2], [7], [5], [6] without using
% cite.sty will become [1], [2], [5]--[7], [9] using cite.sty. cite.sty's
% \cite will automatically add leading space, if needed. Use cite.sty's
% noadjust option (cite.sty V3.8 and later) if you want to turn this off
% such as if a citation ever needs to be enclosed in parenthesis.
% cite.sty is already installed on most LaTeX systems. Be sure and use
% version 5.0 (2009-03-20) and later if using hyperref.sty.
% The latest version can be obtained at:
% http://www.ctan.org/tex-archive/macros/latex/contrib/cite/
% The documentation is contained in the cite.sty file itself.



% *** GRAPHICS RELATED PACKAGES ***
%
\ifCLASSINFOpdf
   \usepackage[pdftex]{graphicx}
 
\else
 
\fi
% graphicx was written by David Carlisle and Sebastian Rahtz. It is
% required if you want graphics, photos, etc. graphicx.sty is already
% installed on most LaTeX systems. The latest version and documentation
% can be obtained at: 
% http://www.ctan.org/tex-archive/macros/latex/required/graphics/
% Another good source of documentation is "Using Imported Graphics in
% LaTeX2e" by Keith Reckdahl which can be found at:
% http://www.ctan.org/tex-archive/info/epslatex/
%
% latex, and pdflatex in dvi mode, support graphics in encapsulated
% postscript (.eps) format. pdflatex in pdf mode supports graphics
% in .pdf, .jpeg, .png and .mps (metapost) formats. Users should ensure
% that all non-photo figures use a vector format (.eps, .pdf, .mps) and
% not a bitmapped formats (.jpeg, .png). IEEE frowns on bitmapped formats
% which can result in "jaggedy"/blurry rendering of lines and letters as
% well as large increases in file sizes.
%
% You can find documentation about the pdfTeX application at:
% http://www.tug.org/applications/pdftex









% *** PDF, URL AND HYPERLINK PACKAGES ***
%
\usepackage{url}
% url.sty was written by Donald Arseneau. It provides better support for
% handling and breaking URLs. url.sty is already installed on most LaTeX
% systems. The latest version and documentation can be obtained at:
% http://www.ctan.org/tex-archive/macros/latex/contrib/url/
% Basically, \url{my_url_here}.




\begin{document}

\title{GambleMania\\ for UTM CSCI 352}
%
%

% received ..."  text while in non-compsoc journals this is reversed. Sigh.

\author{Collin Winstead, Diego Arriaga% <-this % stops a space
}

\IEEEtitleabstractindextext{%
\begin{abstract}
This app allows the user to play three standard casino games: blackjack, three-card poker, and roulette. The player will play against an AI dealer in the blackjack and three-card poker games. In roulette, the program will randomly generate values from the roulette wheel. None of the games in our app will not require any real money to play; all money is virtual and has no monetary value. Users will have their own accounts which will contain login information and the account's balance. This information will be managed with a database.
\end{abstract}

}


% make the title area
\maketitle



\IEEEdisplaynontitleabstractindextext

\IEEEpeerreviewmaketitle



\section{Introduction}


Our project is called GambleMania. It will allow you to play the typical casino games of blackjack, three-card poker, and roulette. The player's balance will be saved in a database along with their login information. We decided to do this project because we thought it would be fun and require a good bit of research and work to finish. As for why we chose the games previously mentioned, it's because they are all fairly simple games to learn and play (probably by design). Because they are fairly simple, they should be relatively easy to program. Our target audience is adults 18 years old and older because the games featured in out app are all typically played in casinos. We expect our audience to use our app simply for fun or to practice blackjack or three-card poker. We did not include roulette in this list because roulette is based solely on randomness, so you can't really practice it.



\subsection{Background}
Minimally, we expect the user to know how to play blackjack, three-card poker, and roulette. However, we could implement a tutorial as a stretch goal to remove this requirement.

\subsection{Impacts}
Although our project is very small and limited, if distributed the project could possibly have a cultural impact by getting more people into the games of blackjack, three-card poker, and roulette. It could also potentially be used for data research and game theory by simulating thousands or millions of blackjack, three-card poker, and roulette games.

\subsection{Challenges}
We foresee the most challenging part of this project as the way the AI dealer will work in the blackjack and three-card poker games. These games have certain rules that need to be checked. For example, in the blackjack game, the dealer must stand on 17. In three-card poker, the player's hand must be checked against the dealer's hand. We think these kinds of obstacles will be our biggest hurdles.


\section{Scope}
This project will be completed when we have implemented a navigable main menu and the games of blackjack, three-card poker, and roulette. The main menu should minimally allow the user to navigate between games by clicking a button (one for each game) and return to desktop by clicking an exit button. The user must be able to play an accurate game of blackjack with the rules of dealer stands on 17 and blackjack pays 3:2. The user must be able to play an accurate game of three-card poker. A fully functioning roulette table must be implemented. Finally, each of the 56 individual cards in a standard deck of playing cards must be designed.


As mentioned earlier in this paper, there are a few stretch goals we can potentially meet should we be able to do so:\\

\begin{enumerate}
\item Create a tutorial for each game for users who do not know how to play.
\item Add more games to the app.
\item Make the main menu and game environments look better.
\item Create a Texas Hold'em style poker game that can be played over a network connection.
\end{enumerate}

\subsection{Requirements}
As mentioned in the Scope section, there are many basic requirements that need to be completed in order for this project to be done. There are also a few stretch goals that we could implement should we get done early. The basis of the basic requirements stems from the rules of play for each of the three basic games we will be implementing. There are also requirements that involve the UI design, some of which are simply required for a functional app and others are for a good-looking app. The stretch requirements are things that would directly make the app better, but are not required for the app to function. In the following subsections, we will describe these requirements in greater detail.

\subsubsection{Functional}
\begin{itemize}

\end{itemize}

\subsubsection{Non-Functional}
\begin{itemize}

\end{itemize}

\subsection{Use Cases}
This section discusses, defines, and ranks the use cases for our project. A table of use cases is shown in Table \ref{tab:useCaseIndex}.

\begin{table}
\centering
\begin{tabular}{|c|c|c|c|c|}
\hline
Use Case ID & Use Case Name & Primary Actor & Complexity & Priority \\
\hline \hline
1 & Play Blackjack & User & Med & 1\\
\hline
2 & Play Three-Card Poker & User & Med & 1\\
\hline
3 & Play Roulette & User & Med & 1\\
\hline
4 & Watch a tutorial & User & Easy & 2\\
\hline
5 & Navigate the Main Menu & User & Easy & 1\\
\hline
6 & Exit the Program & User & Easy & 1\\
\hline
\end{tabular}
\caption{Use Cases}
\label{tab:useCaseIndex}
\end{table}


\subsection{Interface Mockups}
%picture(s) of GUI here


% that's all folks
\end{document}
