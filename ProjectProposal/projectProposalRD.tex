
%% bare_jrnl_compsoc.tex
%% V1.4a
%% 2014/09/17
%% by Michael Shell
%% See:
%% http://www.michaelshell.org/
%% for current contact information.
%%
%% This is a skeleton file demonstrating the use of IEEEtran.cls
%% (requires IEEEtran.cls version 1.8a or later) with an IEEE
%% Computer Society journal paper.
%%
%% Support sites:
%% http://www.michaelshell.org/tex/ieeetran/
%% http://www.ctan.org/tex-archive/macros/latex/contrib/IEEEtran/
%% and
%% http://www.ieee.org/

%%*************************************************************************
%% Legal Notice:
%% This code is offered as-is without any warranty either expressed or
%% implied; without even the implied warranty of MERCHANTABILITY or
%% FITNESS FOR A PARTICULAR PURPOSE! 
%% User assumes all risk.
%% In no event shall IEEE or any contributor to this code be liable for
%% any damages or losses, including, but not limited to, incidental,
%% consequential, or any other damages, resulting from the use or misuse
%% of any information contained here.
%%
%% All comments are the opinions of their respective authors and are not
%% necessarily endorsed by the IEEE.
%%
%% This work is distributed under the LaTeX Project Public License (LPPL)
%% ( http://www.latex-project.org/ ) version 1.3, and may be freely used,
%% distributed and modified. A copy of the LPPL, version 1.3, is included
%% in the base LaTeX documentation of all distributions of LaTeX released
%% 2003/12/01 or later.
%% Retain all contribution notices and credits.
%% ** Modified files should be clearly indicated as such, including  **
%% ** renaming them and changing author support contact information. **
%%
%% File list of work: IEEEtran.cls, IEEEtran_HOWTO.pdf, bare_adv.tex,
%%                    bare_conf.tex, bare_jrnl.tex, bare_conf_compsoc.tex,
%%                    bare_jrnl_compsoc.tex, bare_jrnl_transmag.tex
%%*************************************************************************


% *** Authors should verify (and, if needed, correct) their LaTeX system  ***
% *** with the testflow diagnostic prior to trusting their LaTeX platform ***
% *** with production work. IEEE's font choices and paper sizes can       ***
% *** trigger bugs that do not appear when using other class files.       ***                          ***
% The testflow support page is at:
% http://www.michaelshell.org/tex/testflow/


\documentclass[10pt,conference,onecolumn,compsoc]{IEEEtran}


\usepackage{hyperref}
\usepackage{enumitem}
\setlist[itemize]{leftmargin=3 cm}
\setlist[enumerate]{leftmargin=3cm}



% *** CITATION PACKAGES ***
%
\ifCLASSOPTIONcompsoc
  % IEEE Computer Society needs nocompress option
  % requires cite.sty v4.0 or later (November 2003)
  \usepackage[nocompress]{cite}
\else
  % normal IEEE
  \usepackage{cite}
\fi
% cite.sty was written by Donald Arseneau
% V1.6 and later of IEEEtran pre-defines the format of the cite.sty package
% \cite{} output to follow that of IEEE. Loading the cite package will
% result in citation numbers being automatically sorted and properly
% "compressed/ranged". e.g., [1], [9], [2], [7], [5], [6] without using
% cite.sty will become [1], [2], [5]--[7], [9] using cite.sty. cite.sty's
% \cite will automatically add leading space, if needed. Use cite.sty's
% noadjust option (cite.sty V3.8 and later) if you want to turn this off
% such as if a citation ever needs to be enclosed in parenthesis.
% cite.sty is already installed on most LaTeX systems. Be sure and use
% version 5.0 (2009-03-20) and later if using hyperref.sty.
% The latest version can be obtained at:
% http://www.ctan.org/tex-archive/macros/latex/contrib/cite/
% The documentation is contained in the cite.sty file itself.



% *** GRAPHICS RELATED PACKAGES ***
%
\ifCLASSINFOpdf
   \usepackage[pdftex]{graphicx}
 
\else
 
\fi
% graphicx was written by David Carlisle and Sebastian Rahtz. It is
% required if you want graphics, photos, etc. graphicx.sty is already
% installed on most LaTeX systems. The latest version and documentation
% can be obtained at: 
% http://www.ctan.org/tex-archive/macros/latex/required/graphics/
% Another good source of documentation is "Using Imported Graphics in
% LaTeX2e" by Keith Reckdahl which can be found at:
% http://www.ctan.org/tex-archive/info/epslatex/
%
% latex, and pdflatex in dvi mode, support graphics in encapsulated
% postscript (.eps) format. pdflatex in pdf mode supports graphics
% in .pdf, .jpeg, .png and .mps (metapost) formats. Users should ensure
% that all non-photo figures use a vector format (.eps, .pdf, .mps) and
% not a bitmapped formats (.jpeg, .png). IEEE frowns on bitmapped formats
% which can result in "jaggedy"/blurry rendering of lines and letters as
% well as large increases in file sizes.
%
% You can find documentation about the pdfTeX application at:
% http://www.tug.org/applications/pdftex









% *** PDF, URL AND HYPERLINK PACKAGES ***
%
\usepackage{url}
% url.sty was written by Donald Arseneau. It provides better support for
% handling and breaking URLs. url.sty is already installed on most LaTeX
% systems. The latest version and documentation can be obtained at:
% http://www.ctan.org/tex-archive/macros/latex/contrib/url/
% Basically, \url{my_url_here}.




\begin{document}

\title{Blackjack App\\ for UTM CSCI 352}
%
%

% received ..."  text while in non-compsoc journals this is reversed. Sigh.

\author{Collin Winstead, Jeremy Gordon, Diego Arriaga% <-this % stops a space
}

\IEEEtitleabstractindextext{%
\begin{abstract}
This app will allow you to play a standard game of blackjack against a simple AI dealer. The rules for our blackjack game are as follows: the dealer must stand on 17 and blackjack pays 3:2. No actual money will be required to play this game; the money/chips will be completely virtual. However, since this game is mostly played in casinos, our target audience will be adults 18 years old and older. The app will minimally be played in the console, but we could expand the app outside of it if we are able to do so.
\end{abstract}

}


% make the title area
\maketitle



\IEEEdisplaynontitleabstractindextext

\IEEEpeerreviewmaketitle



\section{Introduction}


For now, our project is simply called Blackjack App. It will allow you to play a typical game of blackjack against an AI dealer. The rules for our blackjack game are as follows: the dealer must stand on 17 and blackjack pays 3:2. No actual money will be required to play this game; the money/chips will be completely virtual. However, since this game is mostly played in casinos, our target audience will be adults 18 years old and older. We are making this app for a few reasons. First, we think it will be a fun project to work on. Second, this app will give people the option to play blackjack without actually gambling any real money, which is great for people who either enjoy the game or want to practice. Finally, this project is flexible in that we can add on to the project should we feel that we will finish too soon, and we could possibly even shift to a more complex game such as three-card poker and reuse a lot of code from the blackjack app.





\subsection{Background}
Minimally, we expect the user to know how to play blackjack. However, we could implement a tutorial as a stretch goal to remove this requirement.

\subsection{Impacts}
Although our project is very small and limited, if distributed the project could possibly have a cultural impact by getting more people into the game of blackjack. It could also potentially be used for data research by simulating thousands or millions of blackjack hands.
\subsection{Challenges}
We forsee the most challenging part of this project as how the dealer AI is going to work. It will likely involve recursively creating Card objects and adding them to a Hand object, which needs to be checked against the player's Hand.\\
Another challenging part of this project would be creating the graphical user interface for the app should we decide to do so.


\section{Scope}
In the worst case (console app), we will be done with this project when the user can accurately play as many blackjack hands as they want to by using console I/O.

In the best case (user interface), we will be done when with this project when the user can accurately play as many blackjack hands as they want through a functional and stylish user interface.\\
As mentioned earlier in this paper, there are a few stretch goals we can potentially meet should we be able to do so:\\

\begin{enumerate}
\item Create a tutorial for users who do not know how to play blackjack.
\item Create a graphical user interface.
\item Only if needed: switch to three-card poker.
\end{enumerate}


% that's all folks
\end{document}


